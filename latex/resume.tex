%%%%%%%%%%%%%%%%%%%%%%%%%%%%%%%%%%%%%%%%%
% One Page Basic Software Developing Engineer Professional CV (North America)
% LaTeX Template
%
% Version 1.0 (10/30/2021)
%
% This template has been downloaded from:
% http://www.LaTeXTemplates.com
%
% Author:
% NíCǎi
%
% Compile note:
% This template requires the resume.cls file to be in the same directory as the
% .tex file. The resume.cls file provides the resume style used for structuring the
% document.
%
%%%%%%%%%%%%%%%%%%%%%%%%%%%%%%%%%%%%%%%%%

%----------------------------------------------------------------------------------------
% PACKAGES,DEPENDENCY AND OTHER DOCUMENT CONFIGURATIONS
%----------------------------------------------------------------------------------------

\documentclass[a4paper]{resume} % Use the custom resume.cls style
\usepackage[left=0.5in,top=0.5in,right=0.5in,bottom=0.5in]{geometry} % Document margins
\usepackage{hyperref}


%----------------------------------------------------------------------------------------
% 0. NAME SECTION
%----------------------------------------------------------------------------------------

\newcommand{\tab}[1]{\hspace{.006\textwidth}\rlap{#1}}
\name{Shawn (Yishun) Jin} % Your name

\address{\small (+1)909-667-1372  ||  yishun.jin@outlook.com || https://www.linkedin.com/in/shawn-jin-80071b165/} 

\begin{document}

%----------------------------------------------------------------------------------------
% 1. EDUCATION SECTION
%----------------------------------------------------------------------------------------

\begin{rSection}{\large Education}

% \small
{\bf University of New South Wales} \hfill { Sep. 2019 - Dec. 2021}
\\Master of Engineering in Information Technology \hfill {\textbf{\underline{Sydney, Australia}}}



{\bf Zhejiang University, Ningbo Institution of Technology} \hfill { Sep. 2015 - Jun. 2019}
\\Bachelor of Engineering in Software Engineering \hfill {\textbf{\underline{(\texttt{GPA:3.8+ Top 5\%}) Zhejiang, China}}}

\end{rSection}


%----------------------------------------------------------------------------------------
% 2. SKILLS SECTION
%----------------------------------------------------------------------------------------


\begin{rSection}{\large Skills}

% \small
\textbf{Programming Languages:}             \hfill  \textbf{Java, Python, C\texttt{\#}, Bash, Perl, Typescript, SQL, HTML, Regex} \\
\textbf{Development \& Tools:}   \hfill  \textbf{Azure, Spring Boot, .NET, Angular, React, Redis, PostgreSQL, MySQL, PowerBI, Microsoft Kusto, Docker, Git, OpenAPI (Swagger), Agile Development (Scrum) }\\

\end{rSection}


%----------------------------------------------------------------------------------------
% 3. EXPERIENCE SECTION
%----------------------------------------------------------------------------------------

\begin{rSection}{\large Experience}
% \small

%job 1 
\begin{rSubsection}{\textbf{\large{Unified Data Catalog Migration}}}{Jan. 2023 - Now}{Microsoft || SOTELS Data Insight Team}{Suzhou. China}

\item Conducted in-depth investigations into \textbf{Kusto} table usage and analyzed data to identify trends and patterns. Collaborated with cross-fabric teams to develop and implement solutions for Unified Data Catalog Migration.
\item Designed and developed a comprehensive dashboard for tracking the migration process with \textbf{Microsoft PowerBI}, enabling the team to monitor progress and identify potential issues in real-time. Ensured the dashboard was user-friendly and accessible to all team members, enhancing transparency and collaboration.
\item Created tools to automate the ingestion of sharepoint online excel data sources with \textbf{Python}, improving efficiency and accuracy while updating Dashboard.

\end{rSubsection}

%Job 2 
\begin{rSubsection}{\textbf{\large{SOTELS {\href{https://www.microsoft.com/en-us/trust-center/privacy/european-data-boundary-eudb}{EUDB (EU Data Boundary)}} Migration}}}{Jan. 2022 - Jan. 2023}{Microsoft || SOTELS EUDB Team}{Suzhou. China}

\item Collaborated with other teams to drive standardization and enforcement efforts across M365 to meet European Compliance \& Privacy regulations. Implemented and maintained data boundary for M365 services.
\item Migrated 20+ \textbf{SCOPE} jobs and over \textbf{1PB} daily data from the current data lake Gen1 to the new data lake Gen2 and also separated EU data to EU environment.
\item Implemented an automation tool using \textbf{Python} to generate View files for downstream consumers, resulting in reducing task workload by over 1 hour per task.
\item Designed and implemented an ad-hoc one-time data copying feature for the EUDB migration portal using \textbf{React} and the \textbf{.NET framework}, enabling seamless data transfer between EU and non-EU environments.

\end{rSubsection}

\begin{rSubsection}{\textbf{\large{Azure DevOps Task Reminder Bot}}}{Aug. 2022}{\href{https://www.credly.com/badges/0a38f67f-44fb-4788-84ff-878430066bc8/linked_in_profile}{Microsoft Global Hackathon 2022}}

\item Designed and implemented an internal \textbf{Teams bot} to remind developers to complete their tasks in Azure DevOps, using technologies such as \textbf{.NET Core}, \textbf{Microsoft Bot Framework}, and \textbf{Microsoft Kusto Storage}.
\item Reduced sprint master workload by over 1 hour per sprint through task status collection automation, allowing team leader to focus on more critical tasks.
\item This Bot has also been introduced to our team weekly sync-up channel.

\end{rSubsection}

%Job 3
\begin{rSubsection}{\textbf{\large{HR++ ( Human Resource Management System )}}}{Oct. 2018 - May. 2019}{Microsoft || SOTELS EUDB Team}{Suzhou. China}

\item Developed a highly customizable HRMS solution for Rococo Ningbo utilizing \textbf{Spring framework} for streamlined code, \textbf{Angular} and \textbf{React} for PC and mobile web front-end, \textbf{Docker} and \textbf{Jenkins} for containerized management and continuous integration. 
\item The solution includes various customizable functional modules, such as fuzzy search, hardware integration, and approval workflows. This resulted in a 50\% faster approval process and a 25\% improvement in search accuracy.
\item Independently developed the resume collection system, resume information extraction, and abstracted a unified interface for extension. This resulted in a 90\% increase in accuracy and a 50\% reduction in processing times.
\item Utilized Yeoman tool to generate simple CRUD template code for the system. This streamlined the development process by eliminating the need for repeated copying and pasting, resulting in a 30\% increase in productivity and a 20\% reduction in development time.

\end{rSubsection}

\end{rSection}

\end{document}

%----------------------------------------------------------------------------------------
% 5. END
%----------------------------------------------------------------------------------------
